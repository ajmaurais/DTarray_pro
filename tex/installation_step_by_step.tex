
\documentclass[12pt]{article}

\newcommand{\VERSION}{1.7.4}

\usepackage[utf8]{inputenc}
\usepackage[english]{babel}
\usepackage[autostyle, english = american]{csquotes}
\MakeOuterQuote{"}

\usepackage{xcolor}
\usepackage{listings, lstautogobble}

\lstset{language=bash,
	keywordstyle=\color{black},
	basicstyle=\small\ttfamily,
	commentstyle=\ttfamily\itshape\color{gray},
	stringstyle=\ttfamily,
	showstringspaces=false,
	breaklines=true,
	frameround=ffff,
	frame=single,
	rulecolor=\color{black},
	autogobble=true
}

\usepackage{hyperref}
\hypersetup{
	colorlinks=true,
	linktoc=all,
	linkcolor=blue,
	citecolor=blue
}

\begin{document}
	
	\title{DTarray\_pro installation and basic usage}
	\author{Aaron Maurais}
	\date{24 September 2018}
	
	\maketitle
	\tableofcontents
	%\newpage
	
	\section{Installation}
	
	\subsection{Download and unpack DTarray\_pro archive file from GitHub}
	\begin{itemize}
		\item Navigate to the \href{https://github.com/ajmaurais/DTarray_pro/releases}{releases} tab on the \texttt{DTarray\_pro} GitHub page.
		
		\item The files for the latest release should be at the top of the page.
		
		\item Download the file: \texttt{Source code (tar.gz)} to you computer for the latest release
		
		\item \texttt{DTarray\_pro} expects to be installed in \texttt{\textasciitilde/local}. The program needs data stored in text files in \texttt{\textasciitilde/local/DTarray\_pro-\VERSION/db} for some features to work. First make the directory \texttt{\textasciitilde/local} on your \texttt{pleiades} account if it doesn't already exist.
		
		\item Transfer the source code archive (should be named something like \texttt{DTarray\_pro-\VERSION.tar}) to your \texttt{pleiades} account using your FTP client of choice.	
		
		\item The source code archive has to be unpacked before you can access it. To unpack the \texttt{.tar} type the following commands in your terminal.
		
		\begin{lstlisting}
			cd ~/local
			tar -xfv DTarray_pro-1.7.4.tar
		\end{lstlisting}
		
		\item As a result, new directory should be created in \texttt{~/local} named \texttt{DTarray\_pro-\VERSION}
		
		\item Once you have unpacked the archive, you no longer need the \texttt{.tar} file and can delete if of you wish.
		
	\end{itemize}

	\subsection{Build DTarray\_pro executable}
	\begin{itemize}
		\item 
	\end{itemize}
	
	
	\section{Usage}
	
\end{document}
